% interactcsesample.tex
% v1.05 - August 2017

\documentclass[]{interact}

\usepackage{epstopdf}% To incorporate .eps illustrations using PDFLaTeX, etc.
\usepackage[caption=false]{subfig}% Support for small, `sub' figures and tables
%\usepackage[nolists,tablesfirst]{endfloat}% To `separate' figures and tables from text if required

%\usepackage[doublespacing]{setspace}% To produce a `double spaced' document if required
%\setlength\parindent{24pt}% To increase paragraph indentation when line spacing is doubled
%\setlength\bibindent{2em}% To increase hanging indent in bibliography when line spacing is doubled

\usepackage{natbib}% Citation support using natbib.sty
\bibpunct[, ]{(}{)}{;}{a}{}{,}% Citation support using natbib.sty
\renewcommand\bibfont{\fontsize{10}{12}\selectfont}% Bibliography support using natbib.sty

\theoremstyle{plain}% Theorem-like structures provided by amsthm.sty
\newtheorem{theorem}{Theorem}[section]
\newtheorem{lemma}[theorem]{Lemma}
\newtheorem{corollary}[theorem]{Corollary}
\newtheorem{proposition}[theorem]{Proposition}

\theoremstyle{definition}
\newtheorem{definition}[theorem]{Definition}
\newtheorem{example}[theorem]{Example}

\theoremstyle{remark}
\newtheorem{remark}{Remark}
\newtheorem{notation}{Notation}

\begin{document}

\articletype{ARTICLE}% Specify the article type or omit as appropriate

\title{The responses of \textit{Spinifex littoreus} to sand burial on the coastal area of Pingtan Island, Fujian Province, South China}

\author{
\name{
  Shuang Song\textsuperscript{a}, 
  Jianhui Du \textsuperscript{a,b} \thanks{CONTACT Jianhui Du. Email: dujh1982@hotmail.com}, 
  Qirui Wu \textsuperscript{a},
  Mingyang Ni \textsuperscript{a} and Yingling Zhang \textsuperscript{a}}\affil{\textsuperscript{a}School of Geography and Planning, Sun Yat-Sen University, Guangzhou, 510275, China; \textsuperscript{b} Guangdong Key Laboratory for Urbanization and Geo-simulation, Guangzhou, 510275, China}
}

\maketitle

\begin{abstract}
\label{abstract}
% 砂生植物对沙埋藏的适应能力对海岸沙丘系统的生态恢复至关重要。
The adaptive capacity of psammophytes to sand burial is crucial for the ecological restoration of coastal dune systems. 
% 研究了福建平潭岛海岸滨鹬对不同埋沙深度和埋沙深度的响应。
The responses of Spinifex littoreus to different sand burial depths and levels were examined on the coast of Pingtan island, Fujian Province, South China. 
% 结果表明,与对照组相比,砂埋对其匍匐茎的垂直生长无显著影响。
The results indicated that, compared to the control group, sand burial on the S. littoreus stolons had no significant impact on its vertical growth of conjoint ramets. 
% 随着砂埋的继续,匍匐茎顶端生长更快,完全砂埋比半砂埋影响更明显。
The stolon apex grew even faster as the sand burial continued, with more obvious influence under complete sand burial than half sand burial. 
% 不同埋沙处理均能刺激和产生荔枝匍匐茎上的不定根,其长度随埋沙水平的变化而变化。
Adventitious roots on S. littoreus stolons were stimulated and produced in all sand burial treatments, the length of which varied according to the sand burial levels. 
% 在不定根和叶片上,三个匍匐茎上的干生物量分配均发生了变化,而在茎上则没有变化。
Dry biomass allocation were altered by sand burial in both adventitious roots and leaves, but not in stems. 
More adventitious roots on base of stolon and leaves on apex of stolon were observed. 
% 通过匍匐茎的快速生长,匍匐茎基部产生大量不定根,匍匐茎顶端的叶片萌发率较高,可以适应完全的砂埋。
S. littoreus can adapt to complete sand burial by rapid growth of stolons, abundant production of adventitious roots on the stolon base, and more germination of leaves on the stolon apex.
\end{abstract}

\begin{keywords}
Psammophytes; sand burial; \textit{Spinifex littoreus}; responses, plant osmotic stress
\end{keywords}


\section{Introduction}

\label{Introduction-1}
% 主要介绍沙生植物对海岸生态系统对重要意义
Sand dunes occupy a finite area in coastal regions, but they are characterized by providing multiple ecological services, and play an important role in the sustainable development in those regions with rising sea levels, surface subsidence and coastal hazards (Martínez et al. 2004, De Battisti \& Griffin 2020). Recently, due to the influence of both climate change and anthropogenic activities, many coastal sand dunes have been modified or destroyed, and this can, or has potentially led to the retreat of coastlines, disappearance of habitats, loss of biodiversity, and severe degradation of ecosystem functions in coastal sand dunes (Feagin et al. 2005, Schlacher et al. 2011, Qu et al. 2017). With the rapid development of economy in China, most of the coastal sand dunes were also eradicated due to real estate and infrastructure development and tourism activities. Almost no well-preserved coastal sand dunes are left, which apart from the loss of ecological functions and habitats, may ultimately result in coastal erosion and the loss of life, property and economy (Yang et al. 2017). Hard engineering structures have been proved to be costly and detrimental for the protection of coastal regions, so it is very important to select natural and environmentally friendly methods for both the protection and restoration of the remaining coastal dune systems (Hanley et al. 2014).

\label{Introduction-2}
% 主要介绍沙埋与沙生植物对关系
Psammophytes can build and stabilize coastal dunes, which will favor the restoration of their ecological functions due to their specific adaptive strategies (Yuan et al. 1993, Nolet et al. 2018, De Battisti \& Griffin 2020). However, only a few species can survive in these ecosystems due to the harsh environmental stress, such as drought, salt spray, strong winds, and especially the frequent and intensive sand burial (Maun \& Lapierre 1986, Hesp 1991, Maun 1994, Maun 1996, Zhao et al. 2014, Du \& Hesp 2020). Sand burial is commonly considered as the selective force for coastal plant regeneration and survival (Moreno-Casasola 1986, Maun 1996, Hwang et al. 2016, Wang et al. 2019), which can decrease and even eliminate the psammophytes if the sand burial levels are over their tolerance capacity, ultimately giving rise to the alteration of species composition in coastal dune systems (Maun \& Perumal 1999, Miller 2015). 

\label{Introduction-3}
% 这一段主要讲海岸沙生植物耐沙埋能力的用处
The tolerance capacity to different sand burial levels is varied among species, which finally affect the initial and subsequent coastal dune development and morphology (Hesp 1989). Generally, the growth of most psammophytes can be expected to be stimulated with low to moderate sand burial (Zhou et al. 2015a, Harris et al. 2017, Brown \& Zinnert 2018, Wang et al. 2019), while intensive sand burial will decrease its survival and per plant biomass (Maun 1996, Franks \& Peterson 2003). Wang et al. (2012) demonstrated that both stem height in the vertical direction and leaf weight of Messerschmidia sibirica were increased with light sand burial, while all decreased with moderate treatments contrasted with the control. In addition, Zhou et al. (2015b) showed that the growth of Artemisia desterorum was enhanced with moderate sand burial, but inhibited with intense sand burial. Elongation of stolons, upward growth of ramets, and development of new adventitious roots are the main adaptive strategies for plants to sand burial, potentially leading to an alteration in the biomass allocation of psammophytes in coastal sand dunes (Dech \& Maun 2006, Frosini et al. 2012, Mendoza-González et al. 2014, Luo et al. 2018). 

\label{Introduction-4}
% 这一段主要介绍老鼠乐与研究去意义
Spinifex littoreus, a herbaceous plant, is the dominant species in coastal foredunes in South China. It often grows vigorously above the backshore, and in the more dynamic foredune, while it declines in the more stabilized backdunes in similarity to its Australian and NZ cousins S. hirsutus and S. sericeus (Hesp 1989, 2002). Its peak growth occurs in summer, with two growth forms, including elongation of horizontal stolons and upward growth of vertical ramets, which sprout out from the same vegetative reproduction, and usually can be recognized as the same ramets (Jackson et al. 1986). The leaf apex of S. littoreus is very sharp and hard, which makes it difficult to be intruded, and is usually recognized as one of the excellent species for sand dune fixation in this region (Yang et al. 2017). Frequent typhoons often occur in the growing season of this species in South China, which causes severe sand burial on S. littoreus, particularly its stolons on nebkhas forming the foredune zone (Yang et al. 2017) (Figure 1). However, the tolerance capacity of this species to sand burial is still unclear at present, and how this species revitalized in the frequent sand burial is less studied. Furthermore, local government has planted an invasive species Casuarinas equisetifolia for the stabilization of nebkhas originally formed by S.littoreus, which causes severe damage to this species, and will inevitably increase the fragility of coastal dune ecosystems in this region (Figure 2). Therefore, understanding the adaptive strategies of S. littoreus to sand burial levels, and its role in the stabilization of coastal dunes are very crucial for the effective coastal management and conservation in South China. 

\begin{figure}
  \centering
  \subfloat[\textit{S. littoreus} under natural sand burial.]{%
  \resizebox*{6cm}{!}{\includegraphics{../figs/sand_buried_plants.jpg}}}\hspace{5pt}
  \subfloat[\textit{C. equisetifolia} danmaged by sand burial.]{%
  \resizebox*{6cm}{!}{\includegraphics{../figs/damaged_trees.jpg}}}
  \caption{(a) Intensive sand burial on the Spinifex littoreus nebkhas on Pingtan island after typhoon Soudelor (1513) landed in Putian City, Fujian Province, China in August, 2015. (b)The nebkhas formed by Spinifex littoreus was stabilized by the planting of an invasive species Casuarinas equisetifolia seedlings on Pingtan Island, Fujian Province, China.} 
  \label{sample-pic}
\end{figure}

\label{Introduction-5}
% 我们的研究假设和研究思路
% 我们认为不同的沙埋情形将改变植株的生长方式,是该植物能够适应台风带来的强烈沙埋的关键性因素。
In this study, we assume that the different sand burial changes the growth process of S. littoreus, which is the key factor for the plants to adapt to the intense sand burial. Then, the vertical height of ramets, horizontal length of stolons, and biomass allocation of adventitious roots, stems and leaves on stolons of S. littoreus under different sand burial levels were simulated and measured in a field experiment, in a growing season on Pingtan Island, Fujian Province, South China.


\section{Material and Methods}
\subsection{Study area}
% 如果手稿要给期刊A4打印了,用 document-class 提供的 largeformat 参数(由 interact.cls)提供
% To prepare a manuscript for a journal that is printed in A4 (two column) format, use the \verb"largeformat" document-class option provided by \texttt{interact.cls}; otherwise the class file produces pages sized for B5 (single column) format by default. The \texttt{geometry} package should not be used to make any further adjustments to the page dimensions.

\subsection{Methods}


\section{Results}

\subsection{Influence of sand burial on the height of vertical ramets}


\subsection{Influence of sand burial on the length of horizontal stolons}


\subsection{Influence of sand burial on the biomass allocation}


\section{Some guidelines for using the standard features of \LaTeX}


\subsection{Tables}

The \texttt{interact} class file will deal with positioning your tables in the same way as standard \LaTeX. It should not normally be necessary to use the optional \texttt{[htb]} location specifiers of the \texttt{table} environment in your manuscript; you may, however, find the \verb"[p]" placement option or the \verb"endfloat" package useful if a journal insists on the need to separate tables from the text.

The \texttt{tabular} environment can be used as shown to create tables with single horizontal rules at the head, foot and elsewhere as appropriate. The captions appear above the tables in the \textsf{Interact} style, therefore the \verb"\tbl" command should be used before the body of the table. For example, Table~\ref{sample-table} is produced using the following commands:
\begin{table}
\tbl{Example of a table showing that its caption is as wide as
 the table itself and justified.}
{\begin{tabular}{lcccccc} \toprule
 & \multicolumn{2}{l}{Type} \\ \cmidrule{2-7}
 Class & One & Two & Three & Four & Five & Six \\ \midrule
 Alpha\textsuperscript{a} & A1 & A2 & A3 & A4 & A5 & A6 \\
 Beta & B2 & B2 & B3 & B4 & B5 & B6 \\
 Gamma & C2 & C2 & C3 & C4 & C5 & C6 \\ \bottomrule
\end{tabular}}
\tabnote{\textsuperscript{a}This footnote shows how to include
 footnotes to a table if required.}
\label{sample-table}
\end{table}
\begin{verbatim}
\begin{table}
\tbl{Example of a table showing that its caption is as wide as
 the table itself and justified.}
{\begin{tabular}{lcccccc} \toprule
 & \multicolumn{2}{l}{Type} \\ \cmidrule{2-7}
 Class & One & Two & Three & Four & Five & Six \\ \midrule
 Alpha\textsuperscript{a} & A1 & A2 & A3 & A4 & A5 & A6 \\
 Beta & B2 & B2 & B3 & B4 & B5 & B6 \\
 Gamma & C2 & C2 & C3 & C4 & C5 & C6 \\ \bottomrule
\end{tabular}}
\tabnote{\textsuperscript{a}This footnote shows how to include
 footnotes to a table if required.}
\label{sample-table}
\end{table}
\end{verbatim}

To ensure that tables are correctly numbered automatically, the \verb"\label" command should be included just before \verb"\end{table}".

The \verb"\toprule", \verb"\midrule", \verb"\bottomrule" and \verb"\cmidrule" commands are those used by \verb"booktabs.sty", which is called by the \texttt{interact} class file and included in the \textsf{Interact} \LaTeX\ bundle for convenience. Tables produced using the standard commands of the \texttt{tabular} environment are also compatible with the \texttt{interact} class file.


\section*{Acknowledgement(s)}

An unnumbered section, e.g.\ \verb"\section*{Acknowledgements}", may be used for thanks, etc.\ if required and included \emph{in the non-anonymous version} before any Notes or References.


\section*{Disclosure statement}

An unnumbered section, e.g.\ \verb"\section*{Disclosure statement}", may be used to declare any potential conflict of interest and included \emph{in the non-anonymous version} before any Notes or References, after any Acknowledgements and before any Funding information.


\section*{Funding}

An unnumbered section, e.g.\ \verb"\section*{Funding}", may be used for grant details, etc.\ if required and included \emph{in the non-anonymous version} before any Notes or References.


\section*{Notes on contributor(s)}

An unnumbered section, e.g.\ \verb"\section*{Notes on contributors}", may be included \emph{in the non-anonymous version} if required. A photograph may be added if requested.


\section*{Nomenclature/Notation}

An unnumbered section, e.g.\ \verb"\section*{Nomenclature}" (or \verb"\section*{Notation}"), may be included if required, before any Notes or References.


\section*{Notes}

An unnumbered `Notes' section may be included before the References (if using the \verb"endnotes" package, use the command \verb"\theendnotes" where the notes are to appear, instead of creating a \verb"\section*").


\section{References}

test\citep{quEffectsSandBurial2017}

\bibliographystyle{tfcse.bst}
\bibliography{interactcsesample}

\end{document}
